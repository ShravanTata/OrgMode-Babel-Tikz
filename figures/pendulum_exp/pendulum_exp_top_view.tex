\documentclass[10pt,tikz]{standalone}

\ifstandalone%
\usepackage{import}%
\import{../../configuration/}{comon_packages.tex}%
\import{../../configuration/}{variables.tex}%
\import{../../configuration/}{conftikz.tex}%
\import{../../configuration/}{custom_config.tex}%
\fi

\begin{document}
\begin{tikzpicture}
  \newcommand{\AxisRotator}[1][rotate=0]{%
    \tikz [x=0.12cm,y=0.30cm,line width=.1ex,-stealth,#1] \draw (0,0) arc (-150:150:1 and 1);%
  }

  % Parameters definitions
  \def\splitw{2} % Width of the split mirrors
  \def\splith{2} % Height of the split mirrors

  \def\photow{1} % Width of the photodiodes
  \def\photoh{3} % Height of the photodiodes

  \def\mirrorw{1} % Width of the mirrors
  \def\mirrorh{1.5} % Height of the mirrors

  \def\pendulumw{3} % Width of the pendulum
  \def\pendulumh{1} % Height of the pendulum

  \def\pendmirrw{1} % Width of the mirror on the pendulum
  \def\pendmirrh{1} % Height of the mirror on the pendulum

  \def\firstinter{3} % First intersection of the beam
  \def\secinter{6}   % Second intersection of the beam
  \def\finalinter{9} % Intersection with the pendulum

  \def\magnetw{1.5} % Width of the magnet
  \def\magneth{1.0} % Height of the magnet

  \def\magnetwb{0.2} % Width of the borders of the magnet
  \def\magnethl{0.2} % Height of the low part of the magnet

  \def\magnetmw{0.3} % Width of the middle part of the magnet
  \def\magnetmh{0.2} % Height of the middle part of the magnet

  \def\magnethg{0.3} % Height of the gap of the magnet
  \def\magnetwg{0.7} % Width of the gap of the magnet

  % Colors
  \definecolor{split}{RGB}{162,255,255} % light blue
  \definecolor{photodiode}{RGB}{254,197,66} % light orange
  \definecolor{mirror}{RGB}{178,178,178} % light grey
  \begin{scope}[rotate=-90]
    % Label positions
    \pgfmathsetmacro{\labelright}{0.5*\splitw+\photow+0.5}%
    \pgfmathsetmacro{\labelleft}{-0.5*\splitw-\photow-0.5}%

    % Laser Source
    \begin{scope}[shift={(0, 0)}]
      \draw[] (-0.5, 0) rectangle node[pos=0.5]{Laser} (0.5, -1.5);
      \draw[] (-0.2, -1.5) to[out=-90,in=45] ++(-1, -1);
      \draw[] ( 0.2, -1.5) to[out=-90,in=45] ++(-1, -1);
    \end{scope}

    % Split Mirror 1
    \begin{scope}[shift={(0, \firstinter)}]
      \draw[fill=split] (-0.5*\splitw, -0.5*\splith) rectangle (0.5*\splitw, 0.5*\splith);
      \draw[dashed] (0.5*\splitw, -0.5*\splith) -- (-0.5*\splitw, 0.5*\splith);
    \end{scope}
    % Photodiode 1
    \begin{scope}[shift={(-0.5*\splitw, \firstinter)}]
      \draw[fill=photodiode] (-\photow, -0.5*\photoh) rectangle node[pos=0.5]{Photodiode} (0, 0.5*\photoh);
      \draw[] (-0.5*\photow, -0.5*\photoh+0.5*\photow) node[]{$\bullet$} to[out=-90,in=0] ++(-\photow, -\photow);
    \end{scope}
    % Mirror 1
    \begin{scope}[shift={(0.5*\splitw, \firstinter)}]
      \draw[fill=mirror] (0, -0.5*\mirrorh) rectangle node[pos=0.5]{Mirror} (\mirrorw, 0.5*\mirrorh);
      \draw[ultra thick] (0, -0.4*\mirrorh) -- (0, 0.4*\mirrorh);
    \end{scope}

    % Split Mirror 2
    \begin{scope}[shift={(0, \secinter)}]
      \draw[fill=split] (-0.5*\splitw, -0.5*\splith) rectangle (0.5*\splitw, 0.5*\splith);
      \draw[dashed] (-0.5*\splitw, -0.5*\splith) -- (0.5*\splitw, 0.5*\splith);
    \end{scope}
    % Photodiode 2
    \begin{scope}[shift={(0.5*\splitw, \secinter)}]
      \draw[fill=photodiode] (0, -0.5*\photoh) rectangle node[pos=0.5]{Photodiode} (\photow, 0.5*\photoh);
      \draw[] (0.5*\photow, -0.5*\photoh+0.5*\photow) node[]{$\bullet$} to[out=-90,in=-180] ++(\photow, -\photow);
    \end{scope}
    % Mirror 2
    \begin{scope}[shift={(-0.5*\splitw, \secinter)}]
      \draw[fill=mirror] (-\mirrorw, -0.5*\mirrorh) rectangle node[pos=0.5]{Mirror} (0, 0.5*\mirrorh);
      \draw[ultra thick] (0, -0.4*\mirrorh) -- (0, 0.4*\mirrorh);
    \end{scope}

    % Pendulum
    \begin{scope}[shift={(0, \finalinter)}]
      % Overall delimitation of the pendulum system
      \coordinate[] (delimmec) at (-0.5*\pendulumw-0.5*\magnetmw-0.5, -1.2);
      \coordinate[] (delimmecbis) at (0.5*\pendulumw+0.5*\magnetmw+0.5, \pendulumh+2.8);
      \draw[dashed] (delimmec) rectangle (delimmecbis);
      \path[] (-0.5*\pendulumw-0.5*\magnetmw-0.5, -1.2) -- (-0.5*\pendulumw-0.5*\magnetmw-0.5, \pendulumh+2.8) node[midway,above]{Mechanical System};

      \draw[] (-0.5*\pendulumw, 0) rectangle (0.5*\pendulumw, \pendulumh);
      \draw[] (-0.5*\pendulumw-0.5, 0.5*\pendulumh) node[]{$\bullet$} -- (0.5*\pendulumw+0.5, 0.5*\pendulumh) node[]{$\bullet$};
      \node[] at (-0.5*\pendulumw-0.3, 0.5*\pendulumh) {\AxisRotator[rotate=-90]};
      \draw[dashed] (0.5*\pendulumw, 0) -- ++(0.5, 0);
      \draw[->, >=latex] (0.5*\pendulumw+0.3, 0) -- ++(0, -0.6) node[below]{$x$};
      \draw[fill=mirror] (-0.5*\pendmirrw, -\pendmirrh) rectangle (0.5*\pendmirrw, 0);
      \draw[ultra thick] (-0.4*\pendmirrw, -\pendmirrh) -- (0.4*\pendmirrw, -\pendmirrh);
    \end{scope}

    % Magnet
    \begin{scope}[shift={(0, \finalinter+\pendulumh)}]
      \draw[] (0, 0) -| ++(0.5*\magnetw, \magneth) -| ++(-0.5*\magnetw+0.5*\magnetwg, -\magnethg) -| (0.5*\magnetw-\magnetwb, \magnethl) -| (-0.5*\magnetw+\magnetwb, \magneth-\magnethg) -| (-0.5*\magnetwg, \magneth) -| (-0.5*\magnetw, 0) -- (cycle);
      % Magnet
      \begin{scope}[shift={(0, \magnethl)}]
        \draw[fill=red]  (-0.5*\magnetmw, 0) rectangle (0.5*\magnetmw, 0.5*\magnetmh);
        \draw[fill=blue] (-0.5*\magnetmw, 0.5*\magnetmh) rectangle (0.5*\magnetmw, \magnetmh);
        % Top conductive Magnet
        \draw[fill=white] (-0.5*\magnetmw, \magnetmh) -| (0.5*\magnetmw, -\magnethl+\magneth-\magnethg) -| ++(0.1, \magnethg) -| ++(-0.2-\magnetmw, -\magnethg) -| (-0.5*\magnetmw, \magnetmh);
        % Force
        \draw[->, >=latex] (0, -0.8*\magnethl+0.8*\magneth)node[]{$\bullet$} -- ++(0, -1.2) node[below]{$F$};
      \end{scope}

      % Coil
      \pgfmathsetmacro{\coilwidth}{0.5*0.5*\magnetmw+0.5*0.1+0.25*\magnetwg}%
      \draw[] ( \coilwidth, \magneth-1.5*\magnethg) -- ++(0, 0.8);
      \draw[] (-\coilwidth, \magneth-1.5*\magnethg) -- ++(0, 0.8);
      % Point on the coil
      \foreach \x in {0,0.1,...,0.8}
      {\node[circle,inner sep=0.6pt,fill] at ( \coilwidth, \x+\magneth-1.5*\magnethg);
        \node[circle,inner sep=0.6pt,fill] at (-\coilwidth, \x+\magneth-1.5*\magnethg);}
      % Actuator Attachement
      \draw[] (-0.5*\pendulumw, \magneth-1.5*\magnethg+0.8) rectangle ++(\pendulumw, \pendulumh);
      % Ground
      \node (ground) [anchor=south, ground, minimum width={\pendulumw cm}, rotate=-90] at (0, \magneth-1.5*\magnethg+0.8+\pendulumh) {};

      % Coil Wires
      \draw[] ( \coilwidth, \magneth-1.5*\magnethg+0.8) node[]{$\bullet$} to[out=90,in=-180] ++(0.6*\pendulumh, 0.6*\pendulumh);
      \draw[] (-\coilwidth, \magneth-1.5*\magnethg+0.8) node[]{$\bullet$} to[out=90,in=-180] ++(0.6*\pendulumh, 0.6*\pendulumh);
    \end{scope}

    % LASER
    \draw[red, ->-=.5] (0, 0) -- (0, \firstinter);
    \draw[red, ->-=.6, -<-=.4] (0, \firstinter) -- (-, \secinter);
    \draw[red, ->-=.7, -<-=.3] (0, \secinter) -- (0, \finalinter-\pendmirrh);

    \draw[red, ->-=.7, -<-=.3] (0, \firstinter) -- ++( 0.5*\splitw, 0);
    \draw[red, ->-=.5] (0, \firstinter) -- ++(-0.5*\splitw, 0);
    \draw[red, ->-=.5] (0, \secinter)   -- ++( 0.5*\splitw, 0);
    \draw[red, ->-=.7, -<-=.3] (0, \secinter)   -- ++(-0.5*\splitw, 0);

    % Delimitation of the Interferometer system
    \coordinate[] (deliminter) at ($(delimmec)-(0, 0.1)$);
    \coordinate[] (deliminterbis) at ($(delimmecbis)-(0, 15.4)$);
    \draw[dashed] (deliminter) rectangle (deliminterbis);
    \path[] (deliminter) -- (deliminterbis -| deliminter) node[midway, above]{Interferometer};
  \end{scope}

\end{tikzpicture}
\end{document}