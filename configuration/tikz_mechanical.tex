% =============================================================
% Tikz Style for mechanical figures
% =============================================================

% =============================================================
% Spring
% =============================================================
\tikzset{%
  spring/.style={%
    thick,
    decoration={
      zigzag,
      pre length  = #1cm,
      post length = #1cm,
      segment length = 6
    },
    decorate
  },
  spring/.default={0.2}
}
% =============================================================


% =============================================================
% Coil
% =============================================================
\tikzset{%
  coil/.style n args={2}{%
    thick,
    decoration={
      coil,
      pre length  = #1cm,
      post length = #2cm,
      segment length = 4
    },
    decorate
  },
  coil/.default={0.3}{0.3}
}
% =============================================================


% =============================================================
% Damper
% =============================================================
\tikzset{%
  damper/.style n args={2}{%
    thick,
    decoration={markings, mark connection node=dmp, mark=at position 0.5 with {
        \node (dmp) [thick,
                     inner sep = 0pt,
                     transform shape,
                     rotate  =-90,
                     minimum width  = #1pt,
                     minimum height = #2pt,
                     draw=none] {};
        \draw [thick] ($(dmp.north east)+(0.6*#2pt,0)$) -- (dmp.south east) -- (dmp.south west) -- ($(dmp.north west)+(0.6*#2pt,0)$);
        \draw [thick] ($(dmp.north)+(0,-0.3*#1pt)$) -- ($(dmp.north)+(0,0.3*#1pt)$);
      }
    },
    decorate
  },
  damper/.default={12}{3}
}
% =============================================================


% =============================================================
% Actuator
% =============================================================
\tikzset{%
  actuator/.style n args={2}{%
    thick,
    draw=none,
    decoration={
      markings,
      mark connection node=my node,
      mark=at position .5 with {
        \node [draw, inner sep=0pt, minimum width=#1cm, minimum height=#2cm,
        transform shape] (my node) {};
      },
      mark=at position .0 with {
        \draw[<-] (0, 0) -- (my node);
      },
      mark=at position 1.0 with {
        \draw[<-] (0, 0) -- (my node);
      }
    },
    decorate
  },
  actuator/.default={0.2}{0.5}
}
% =============================================================


% =============================================================
% Ground
% =============================================================
\tikzset{%
  ground/.style n args={2}{%
    fill,
    pattern = north east lines,
    draw = none,
    anchor = north,
    minimum width  = #1cm,
    minimum height = #2cm,
    append after command={
      (\tikzlastnode.north west) edge (\tikzlastnode.north east)
    }
  },
  ground/.default={2.5}{0.3}
}
% =============================================================


% =============================================================
% Force Sensor
% =============================================================
\tikzset{%
  forcesensor/.style n args={2}{%
    rectangle,
    outer sep=0pt,
    inner sep=0pt,
    draw=black,
    fill=white!60!black,
    anchor=south,
    minimum width =#1cm,
    minimum height=#2cm,
    append after command={
      [every edge/.append style={
        thick,
        black,
      }]
      (\tikzlastnode.north west) edge (\tikzlastnode.south east)
      (\tikzlastnode.north east) edge (\tikzlastnode.south west)
    }
  },
  forcesensor/.default={2.0}{0.5}
}
% =============================================================


% =============================================================
% Inertial Sensor
% =============================================================
\tikzset{%
  inertialsensor/.style n args={1}{%
    rectangle,
    outer sep=0pt,
    inner sep=0pt,
    draw=black,
    fill=white!60!black,
    anchor=south east,
    minimum width=#1cm,
    minimum height=#1cm,
    append after command={
      [every edge/.append style={
        thick,
        black,
      }]
      (\tikzlastnode.north west) edge (\tikzlastnode.south east)
      (\tikzlastnode.north east) edge (\tikzlastnode.south west)
    }
  },
  forcesensor/.default={0.5}
}
% =============================================================


\newcommand{\AxisRotator}[1][rotate=0]{%
  \tikz [x=0.1cm,y=0.30cm,-stealth,#1] \draw (0,0) arc (-150:150:1 and 1);%
}
\tikzstyle{cross}=[path picture={
  \draw[black]
  (path picture bounding box.south east) -- (path picture bounding box.north west) (path picture bounding box.south west) -- (path picture bounding box.north east);
}]

% =============================================================
% Voice coil macro
% =============================================================
\def\voicecoil#1#2#3{
  % ======================
  % Parameters
  % ======================
  \def\voicecoilw{#1} % Total Width
  \def\voicecoilh{#2} % Total Height

  \def\magnetw{\voicecoilw} % Width of the magnet
  \def\magneth{\voicecoilh/1.4} % Height of the magnet

  \def\magnetwb{0.15*\magnetw} % Width of the borders of the magnet
  \def\magnetmw{0.15*\magnetw} % Width of the middle part of the magnet
  \def\magnetwg{0.5*\magnetw} % Width of the gap of the magnet

  \def\magnethl{\magnetwb} % Height of the low part of the magnet
  \def\magnetmh{0.15*\magneth} % Height of the middle part of the magnet
  \def\magnethg{0.2*\magneth} % Height of the gap of the magnet
  % ======================

  \begin{scope}[shift={(0.5*\voicecoilw, 0.5*\voicecoilh)}, rotate=#3, shift={(0, -0.5*\voicecoilh)}]
    % ======================
    % Magnet
    % ======================
    \draw[] (0, 0) -| ++(0.5*\magnetw, \magneth) -| ++(-0.5*\magnetw+0.5*\magnetwg, -\magnethg) -| (0.5*\magnetw-\magnetwb, \magnethl) -| (-0.5*\magnetw+\magnetwb, \magneth-\magnethg) -| (-0.5*\magnetwg, \magneth) -| (-0.5*\magnetw, 0) -- (cycle);
    \begin{scope}[shift={(0, \magnethl)}]
      \draw[fill=red]  (-0.5*\magnetmw, 0) rectangle (0.5*\magnetmw, \magnetmh);
      \draw[fill=blue] (-0.5*\magnetmw, \magnetmh) rectangle (0.5*\magnetmw, 2*\magnetmh);
      % Top conductive Magnet
      \draw[fill=white] (-0.5*\magnetmw, 2*\magnetmh) -| (0.5*\magnetmw, -\magnethl+\magneth-\magnethg) -| ++(0.1, \magnethg) -| ++(-0.2-\magnetmw, -\magnethg) -| (-0.5*\magnetmw, \magnetmh);
    \end{scope}
    % ======================

    % ======================
    % Coil
    % ======================
    \pgfmathsetmacro{\coilwidth}{0.5*0.5*\magnetmw+0.5*0.1+0.25*\magnetwg}%
    \draw[] ( \coilwidth, 0.5*\magneth) -- ++(0, 0.7*\magneth);
    \draw[] (-\coilwidth, 0.5*\magneth) -- ++(0, 0.7*\magneth);
    % Point on the coil
    \foreach \x in {0,1,...,9}
    {
      \node[circle,inner sep=0.6pt,fill] at ( \coilwidth, \x*0.7*\magneth/10+0.5*\magneth);
      \node[circle,inner sep=0.6pt,fill] at (-\coilwidth, \x*0.7*\magneth/10+0.5*\magneth);
    }
    \draw[] (-0.5*\magnetw, 1.2*\magneth) rectangle ++(\magnetw, \magnethg);
    % ======================

    % ======================
    % Coordinates
    % ======================
    % Force
    \coordinate[] (vc_force) at (0, \magneth-0.5*\magnethg);
    % Coil
    \coordinate[] (vc_coil) at (0, \voicecoilh);
    % Magnet
    \coordinate[] (vc_magnet) at (0, 0);
    % Coil Wires
    \coordinate[] (vc_wire_one) at ( \coilwidth, 1.2*\magneth);
    \coordinate[] (vc_wire_two) at (-\coilwidth, 1.2*\magneth);
    % ======================
  \end{scope}
}
% =============================================================