% =============================================================
% Packages
% =============================================================
\usepackage{tikz}             % Tikz
\usepackage{tikzscale}        % Used to scale Tikz graphics
\usepackage{adjustbox}        % Used to proper positioning of tikz pictures
\usepackage{circuitikz}       % Draw electronic circuits
\usepackage{pgfpages}         % Needed to use notes
\usepackage{pgfplots}         % Used to plot functions
% =============================================================


% =============================================================
% Tikz Libraries
% =============================================================
\usetikzlibrary{arrows}                   % Arrow tip library
\usetikzlibrary{calc}                     % The library allows advanced Coordinate Calculations
\usetikzlibrary{intersections}            % calculate intersections of paths
\usetikzlibrary{matrix}                   %
\usetikzlibrary{patterns}                 %
\usetikzlibrary{shapes}                   % Defines circle and rectangle
\usetikzlibrary{shapes.geometric}         % Use for the shape diamond and isosceles triangle
\usetikzlibrary{snakes}                   % snake=coil and snake=zigzag using segment amplitude=10pt
\usetikzlibrary{positioning}              % Additional options for placing nodes
\usetikzlibrary{3d}                       % Plot 3D shapes
\usetikzlibrary{spy}                      % Creating a magnified area
\usetikzlibrary{decorations.text}         % Used to make text follows a curve
\usetikzlibrary{decorations.pathmorphing} % deformation of a path
\usetikzlibrary{decorations.markings}     % Used for spring and damper
\usetikzlibrary{babel}                    % A tiny library that make the interaction with the babel package easier
\usetikzlibrary{plotmarks}                % This library defines a number of plot marks
\usetikzlibrary{fit}                      % Used to make rectangle as nodes by specifying two points
% =============================================================


% =============================================================
% PGF Plot libraries and config
% =============================================================
\usepgfplotslibrary{patchplots}

\pgfplotsset{compat=newest}
\pgfplotsset{plot coordinates/math parser=false}
% =============================================================


% =============================================================
% Setup size of figures
% =============================================================
\newlength{\fheight}%
\newlength{\fwidth}%

\setlength{\fwidth}{85mm}
\setlength{\fheight}{112mm}
% =============================================================


% =============================================================
% Setup Arrows style
% =============================================================
% \tikzset{>=latex}
\tikzset{every path/.style={line width=1pt}}
\tikzset{>=stealth}
% =============================================================

% =============================================================
% Tikz for diagram control
% =============================================================
\tikzstyle{dot} = [draw, circle, fill, minimum size=2pt, inner sep=0pt, outer sep=0pt]
\tikzstyle{DAC} = [signal]
\tikzstyle{ADC} = [signal, signal to=west]
\tikzstyle{block} = [minimum width=\blockw, minimum height=\blockh]
% =============================================================

% =============================================================
% Tikz Style for mechanical figures
% =============================================================
\tikzstyle{spring}=[thick,decorate,decoration={zigzag,pre length=0.3cm,post length=0.3cm,segment length=6}]
\tikzstyle{coil}=[thick,decorate,decoration={coil,pre length=0.3cm,post length=0.3cm,segment length=4}]
\tikzstyle{damper}=[thick,decoration={markings, mark connection node=dmp, mark=at position 0.5 with {
  \node (dmp) [thick,inner sep=0pt,transform shape,rotate=-90,minimum width=15pt,minimum height=3pt,draw=none] {};
  \draw [thick] ($(dmp.north east)+(2pt,0)$) -- (dmp.south east) -- (dmp.south west) -- ($(dmp.north west)+(2pt,0)$);
  \draw [thick] ($(dmp.north)+(0,-5pt)$) -- ($(dmp.north)+(0,5pt)$);
  }
}, decorate]
\tikzstyle{actuator}=[thick, draw=none, decoration={markings,
  mark connection node=my node,
  mark=at position .5 with {
  \node [draw, minimum width=5pt, minimum height=20pt] (my node) {};
  },
  mark=at position .0 with {
  \draw[<-] (0, 0) -- (my node.south);
  },
  mark=at position 1.0 with {
  \draw[<-] (0, 0) -- (my node.north);
  }
},decorate]
\tikzstyle{ground}=[fill,pattern=north east lines,draw=none,minimum width=0.75cm,minimum height=0.3cm]
% =============================================================


% =============================================================
% Tikz Style for Optical setups
% =============================================================
\tikzset{->-/.style={decoration={
      markings,
      mark=at position #1 with {\arrow{>}}},postaction={decorate}}}
\tikzset{-<-/.style={decoration={
      markings,
      mark=at position #1 with {\arrow{<}}},postaction={decorate}}}
% =============================================================


% =============================================================
% Voice coil macro
% =============================================================
\def\voicecoil#1#2#3{
  % ======================
  % Parameters
  % ======================
  \def\voicecoilw{#1} % Total Width
  \def\voicecoilh{#2} % Total Height

  \def\magnetw{\voicecoilw} % Width of the magnet
  \def\magneth{\voicecoilh/1.4} % Height of the magnet

  \def\magnetwb{0.15*\magnetw} % Width of the borders of the magnet
  \def\magnetmw{0.15*\magnetw} % Width of the middle part of the magnet
  \def\magnetwg{0.5*\magnetw} % Width of the gap of the magnet

  \def\magnethl{\magnetwb} % Height of the low part of the magnet
  \def\magnetmh{0.15*\magneth} % Height of the middle part of the magnet
  \def\magnethg{0.2*\magneth} % Height of the gap of the magnet
  % ======================

  \begin{scope}[shift={(0.5*\voicecoilw, 0.5*\voicecoilh)}, rotate=#3, shift={(0, -0.5*\voicecoilh)}]
    % ======================
    % Magnet
    % ======================
    \draw[] (0, 0) -| ++(0.5*\magnetw, \magneth) -| ++(-0.5*\magnetw+0.5*\magnetwg, -\magnethg) -| (0.5*\magnetw-\magnetwb, \magnethl) -| (-0.5*\magnetw+\magnetwb, \magneth-\magnethg) -| (-0.5*\magnetwg, \magneth) -| (-0.5*\magnetw, 0) -- (cycle);
    \begin{scope}[shift={(0, \magnethl)}]
      \draw[fill=red]  (-0.5*\magnetmw, 0) rectangle (0.5*\magnetmw, \magnetmh);
      \draw[fill=blue] (-0.5*\magnetmw, \magnetmh) rectangle (0.5*\magnetmw, 2*\magnetmh);
      % Top conductive Magnet
      \draw[fill=white] (-0.5*\magnetmw, 2*\magnetmh) -| (0.5*\magnetmw, -\magnethl+\magneth-\magnethg) -| ++(0.1, \magnethg) -| ++(-0.2-\magnetmw, -\magnethg) -| (-0.5*\magnetmw, \magnetmh);
    \end{scope}
    % ======================

    % ======================
    % Coil
    % ======================
    \pgfmathsetmacro{\coilwidth}{0.5*0.5*\magnetmw+0.5*0.1+0.25*\magnetwg}%
    \draw[] ( \coilwidth, 0.5*\magneth) -- ++(0, 0.7*\magneth);
    \draw[] (-\coilwidth, 0.5*\magneth) -- ++(0, 0.7*\magneth);
    % Point on the coil
    \foreach \x in {0,1,...,9}
    {
      \node[circle,inner sep=0.6pt,fill] at ( \coilwidth, \x*0.7*\magneth/10+0.5*\magneth);
      \node[circle,inner sep=0.6pt,fill] at (-\coilwidth, \x*0.7*\magneth/10+0.5*\magneth);
    }
    \draw[] (-0.5*\magnetw, 1.2*\magneth) rectangle ++(\magnetw, \magnethg);
    % ======================
  \end{scope}

  % ======================
  % Coordinates
  % ======================
  % Force
  \coordinate[] (vc_force) at (0, \magneth-0.5*\magnethg);
  % Coil
  \coordinate[] (vc_coil) at (0, \voicecoilh);
  % Magnet
  \coordinate[] (vc_magnet) at (0, 0);
  % Coil Wires
  \coordinate[] (vc_wire_one) at ( \coilwidth, 1.2*\magneth);
  \coordinate[] (vc_wire_two) at (-\coilwidth, 1.2*\magneth);
  % ======================
}
% =============================================================
